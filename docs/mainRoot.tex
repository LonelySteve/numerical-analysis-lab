\documentclass{ctexart}

% 用于导入图片
% https://tex.stackexchange.com/a/23076
\usepackage{graphicx}
% 用于页面设置
\usepackage[a4paper]{geometry}
% 用于获取标题,作者等信息
\usepackage{titling}
% 用于下划线
\usepackage{ulem}
% 用于计算文字宽度
\usepackage{calc}
% 用于获得灰度模型
\usepackage{xcolor}
% 用于数学公式
\usepackage{amsmath}
\usepackage{lstfiracode} % https://ctan.org/pkg/lstfiracode
% 设置代码块风格
\lstset{
  language=Matlab, 
  numbers=left,
  backgroundcolor=\color{lightgray!40!white},
  basicstyle=\ttfamily,   % Use \ttfamily for source code listings
  keywordstyle=\color{blue!70}\bfseries,
  commentstyle=\ttfamily\color{green!40!black},
}
% 设置页面大小
\geometry{hmargin=3.17cm, vmargin=2.54cm}
\title{《数值分析》\\ 实\hspace{0.5cm}验\hspace{0.5cm}报\hspace{0.5cm}告}
\author{\pyvar{author_info.author_name.value}}
% 行间距
\linespread{1.3}

\ctexset{
    section={
        name={实验},
        format=\centering\kaishu\zihao{-2},
        number=\chinese{section},
    },
    subsection={
        name={,、},
        format=\kaishu\zihao{-4},
        number=\chinese{subsection},
    },
    subsubsection={
        name={,、},
        format=\kaishu\zihao{5},
        number=\arabic{subsubsection}
    }
}

% 重新定义标题页
\renewcommand{\maketitle}{
    \begin{titlepage}
        \begin{center}
            \includegraphics[width=2.84cm]{\pyvar{"imgs/schoolLogo1.png" | real_relative_path}}
        \end{center}
        \begin{center}
            \includegraphics[width=6.67cm]{\pyvar{"imgs/schoolLogo2.png" | real_relative_path}}
        \end{center}
        \zihao{-0}
        \heiti
        % 标题
        \begin{center}
            \thetitle
        \end{center}
        \vspace{1cm}

        \kaishu
        \zihao{2}

        \newlength{\underlineLength}
        \setlength{\underlineLength}{8cm}
        \newlength{\lrSpaceLength}

        %% for item in  ["author_name", "author_student_number", "author_major_name", "author_class_name"]
        \setlength{\lrSpaceLength}{
            (\underlineLength - \widthof{\pyvar{author_info[item].value}}) / 2
        }
        % \pyvar{author_info[item]["trans"]}
        \begin{center}
            \pyvar{author_info[item].trans}:\uline{\hspace{\lrSpaceLength}\pyvar{author_info[item].value}\hspace{\lrSpaceLength}}
        \end{center}
        %% endfor
        \vspace{1cm}
        \begin{center}
            \selectcolormodel{gray}
            \includegraphics[width=2.33cm]{\pyvar{"imgs/majorLogoGray.png" | real_relative_path}}
        \end{center}
        \begin{center}
            \zihao{-1}
            计算机科学学院 \\
            \vspace{1cm}
            \zihao{-2}
            2019-2020学年 第2学期
        \end{center}
    \end{titlepage}
}


\begin{document}
% 标题页
\maketitle

\section{}

\zihao{-5}

\subsection{问题的描述}


\subsubsection{设线性方程组}



\[
    \begin{bmatrix}
        4  & 2  & -3  & -1 & 2  & 1   & 0  & 0  & 0  & 0  \\
        8  & 6  & -5  & -3 & 6  & 5   & 0  & 1  & 0  & 0  \\
        4  & 2  & -2  & -1 & 3  & 2   & -1 & 0  & 3  & 1  \\
        0  & -2 & 1   & 5  & -1 & 3   & -1 & 1  & 9  & 4  \\
        -4 & 2  & 6   & -1 & 6  & 7   & -3 & 3  & 2  & 3  \\
        8  & 6  & -8  & 5  & 7  & 17  & 2  & 6  & -3 & 5  \\
        0  & 2  & -1  & 3  & -4 & 2   & 5  & 3  & 0  & 1  \\
        16 & 10 & -11 & -9 & 17 & 34  & 2  & -1 & 2  & 2  \\
        4  & 6  & 2   & -7 & 13 & 9   & 2  & 0  & 12 & 4  \\
        0  & 0  & -1  & 8  & -3 & -24 & -8 & 6  & 3  & -1
    \end{bmatrix}
    \begin{bmatrix}
        x_1 \\
        x_2 \\
        x_3 \\
        x_4 \\
        x_5 \\
        x_6 \\
        x_7 \\
        x_8 \\
        x_9 \\
        x_{10}
    \end{bmatrix}
    =
    \begin{bmatrix}
        5  \\
        12 \\
        3  \\
        2  \\
        3  \\
        46 \\
        13 \\
        38 \\
        19 \\
        -21
    \end{bmatrix}
\]

\[x^* = (1, -1, 0, 1, 2, 0, 3, 1, -1, 2) ^ T\]

利用 Gauss 顺序消去法或 Gauss 列主元消去法求解。

\subsubsection{设对称正定阵系数阵线方程组}



\[
    \begin{bmatrix}
        4  & 2  & -4 & 0  & 2   & 4  & 0   & 0  \\
        2  & 2  & -1 & -2 & 1   & 3  & 2   & 0  \\
        -4 & -1 & 14 & 1  & -8  & -3 & 5   & 6  \\
        0  & -2 & 1  & 6  & -1  & -4 & -3  & 3  \\
        2  & 1  & -8 & -1 & 22  & 4  & -10 & -3 \\
        4  & 3  & -3 & -4 & 4   & 11 & 1   & -4 \\
        0  & 2  & 5  & -3 & -10 & 1  & 14  & 2  \\
        0  & 0  & 6  & 3  & -3  & -4 & 2   & 19
    \end{bmatrix}
    \begin{bmatrix}
        x_1 \\
        x_2 \\
        x_3 \\
        x_4 \\
        x_5 \\
        x_6 \\
        x_7 \\
        x_8
    \end{bmatrix}
    =
    \begin{bmatrix}
        0   \\
        -6  \\
        20  \\
        23  \\
        9   \\
        -22 \\
        -15 \\
        45
    \end{bmatrix}
\]

\[
    x^* = (1, -1, 0, 2, 1, -1, 0, 2)^T
\]

利用平方根法或改进平方根法求解。

\subsubsection{三对角形线性方程组}



\[
    \begin{bmatrix}
        4  & -1 & 0  & 0  & 0  & 0  & 0  & 0  & 0  & 0  \\
        -1 & 4  & -1 & 0  & 0  & 0  & 0  & 0  & 0  & 0  \\
        0  & -1 & 4  & -1 & 0  & 0  & 0  & 0  & 0  & 0  \\
        0  & 0  & -1 & 4  & -1 & 0  & 0  & 0  & 0  & 0  \\
        0  & 0  & 0  & -1 & 4  & -1 & 0  & 0  & 0  & 0  \\
        0  & 0  & 0  & 0  & -1 & 4  & -1 & 0  & 0  & 0  \\
        0  & 0  & 0  & 0  & 0  & -1 & 4  & -1 & 0  & 0  \\
        0  & 0  & 0  & 0  & 0  & 0  & -1 & 4  & -1 & 0  \\
        0  & 0  & 0  & 0  & 0  & 0  & 0  & -1 & 4  & -1 \\
        0  & 0  & 0  & 0  & 0  & 0  & 0  & 0  & -1 & 4
    \end{bmatrix}
    \begin{bmatrix}
        x_1 \\
        x_2 \\
        x_3 \\
        x_4 \\
        x_5 \\
        x_6 \\
        x_7 \\
        x_8 \\
        x_9 \\
        x_{10}
    \end{bmatrix}
    =
    \begin{bmatrix}
        7   \\
        5   \\
        -13 \\
        2   \\
        6   \\
        -12 \\
        -14 \\
        -4  \\
        5   \\
        -5
    \end{bmatrix}
\]

利用追赶法求解。}

\subsection{方法描述}

\subsubsection{高斯消去法}

高斯消去法,该方法以数学家卡尔·高斯命名,但最早出现于中国古籍《九章算术》,成书于约公元前150年。

整体的思路是,将线性方程组的增广矩阵通过初等行变换得到行阶梯形矩阵,再由下至上回代计算得到线性方程组的解。

初等行变换有三种方式:

\begin{enumerate}
    \item 交换某两行
    \item 将某行乘以一个非零的数
    \item 将某行的倍数加到另一行
\end{enumerate}

在人工进行初等行变换时,可以配合使用以上三种方法来达到计算行阶梯形矩阵的目的,在计算机程序中实现时,可以仅使用最后一种方式来完成行阶梯形矩阵的计算。


\subsubsection{平方根法}

平方根法,也称为 choleskly 方法,用于将给定的系数矩阵 \(A\) 分解为
\(A = L L^T\) 的形式。这里的系数矩阵 \(A\) 是对称正定矩阵,其中的 \(L\)
是下三角矩阵。

由此可以写出以下等式: 

\begin{equation*}
    \begin{aligned}
        A = LL^T & =
        \begin{bmatrix}
            l_{11} & 0      & 0      \\
            l_{21} & l_{22} & 0      \\
            l_{31} & l_{32} & l_{33} \\
        \end{bmatrix}
        \begin{bmatrix}
            l_{11} & l_{21} & l_{31} \\
            0      & l_{22} & l_{32} \\
            0      & 0      & l_{33} \\
        \end{bmatrix} \\
                      & =
        \begin{bmatrix}
            l_{11}^2     &                             & (symmetric)                     \\
            l_{21}l_{11} & l_{21}^2 + l_{22}^2                                           \\
            l_{31}l_{11} & l_{31}l_{21} + l_{32}l_{22} & l_{31}^2 + l_{32}^2 + l_{33} ^2 \\
        \end{bmatrix}
    \end{aligned}
\end{equation*}

由此可以得到:

\begin{equation*}
    L =
    \begin{bmatrix}
        (\pm) \sqrt{a_{11}}   & 0                            & 0                               \\
        a_{21} / l_{11} &  (\pm)\sqrt{a_{22}-l_{21}^2}       & 0                               \\
        a_{31} / l_{11} & (a_{32}-l_{31}l_{21})/l_{22} &  (\pm) \sqrt{a_{33}-l_{31}^2-l_{32}^2} \\
    \end{bmatrix}
\end{equation*}

根据规律可以归纳得到:

\begin{equation*}
    l_{j,j} = (\pm) \sqrt{a_{j,j} - \sum_{k=1}^{j-1} l_{j,k}^2} \\
\end{equation*}

\[
    l_{i,j} = (a_{i,j} - \sum_{k=1}^{j-1}l_{i,k}l_{j,k}) / l_{j,j},  i > j
\]

因此,要计算${l}_{i,j}(i\neq j)$ 只需利用矩阵 $L$ 的左、上方元素的值。计算通常是以下面其中一种顺序进行的。\footnote{https://zh.wikipedia.org/wiki/Cholesky}

\begin{enumerate}
    \item Cholesky–Banachiewicz:算法从矩阵L的左上角开始,依行进行计算。
    \item Cholesky–Crout:算法从矩阵L的左上角开始,依列进行计算。
\end{enumerate}

若有需要,整个矩阵可以逐个元素计算得出,无论使用何种顺序读取。

将系数矩阵 \(A\) 分解为 \(A = L L^T\) 后,由 \(Ax = b\)
可得:\(LL^Tx = b\)。

如果进行以下分解:

\[
    \left\{\begin{matrix}
        y = L^Tx \\
        Ly = b
    \end{matrix}\right.
\]


就可以先计算 \(y\),再计算 \(x\),因为 \(L\) 是下三角矩阵,而 \(L^T\)
是上三角矩阵,这样会比直接使用 \(A\) 计算更简便。

\subsubsection{}
}

\subsection{算法设计}

\subsubsection{问题一}

\begin{breakablealgorithm}
    \caption{高斯顺序消去法}
    \begin{algorithmic}[1]
        \REQUIRE{矩阵 $A$ 的对角线元素均不为 0}
        \STATE $B \leftarrow [A, b]$ \COMMENT{通过合并系数矩阵 $A$ 和常数矩阵 $b$ 获得增广矩阵 $B$ }
        \STATE $[n, m] \leftarrow size(B)$ \COMMENT{求增广矩阵 B 的行数和列数,分别记做 n 和 m}
        \FOR{$k=1$ \TO $n$}
        \FOR{$o=k+1$ \TO $n$}
        \FOR{$p=0$ \TO $m$}
        \STATE $B_{o, p} = B_{o, p} - (B_{o, k} / B_{k, k}) * B_{k, p}$
        \ENDFOR
        \ENDFOR
        \ENDFOR
        \STATE $x \leftarrow zeros(m-1, 1)$ \COMMENT{分配 x 的空间}
        \FOR{$k=n$ \TO $1$ by step $-1$}
        \FOR{$q=k+1$ \TO $m-1$}
        \STATE $x_k \leftarrow (B_{k, m} - B_{k, q} * x_q ) / B_{k, k}$
        \ENDFOR
        \ENDFOR
    \end{algorithmic}
\end{breakablealgorithm}

\subsubsection{问题二}

\begin{breakablealgorithm}
    \caption{cholesky 算法}
    \begin{algorithmic}[1]
        \REQUIRE{被分解的矩阵必须为厄米特矩阵或符合正定矩阵定义的矩阵}
        \STATE $[n, ~] \leftarrow size(A)$ \COMMENT{获取矩阵 A 的行数 n}
        \STATE $L \leftarrow zeros(n, n)$ \COMMENT{分配矩阵 L 的空间}
        \FOR{$j=1$ \TO $n$}
        \FOR{$i=j$ \TO $n$}
        \IF{$i == j$}
        \STATE $L_{i, j} \leftarrow \sqrt{A_{i, j}} - \sum_{k=1}^{j-1}L_{j, k}^2$
        \ELSE
        \STATE $L_{i, j} \leftarrow (A_{i, j} - \sum_{k=1}^{j-1}L_{i, k}L_{j, k}) / L_{j, j}$
        \ENDIF
        \ENDFOR
        \ENDFOR
    \end{algorithmic}
\end{breakablealgorithm}

\subsubsection{问题三}
}

\subsection{程序代码}

\subsubsection{问题一}
\subsubsection{问题二}
%% for name in ["cholesky", "squareRoot"]
\textit{\pyvar{name}}
\begin{lstlisting}[frame=single]
\pyblock{set rel_path = name+".m"}
\pyvar{rel_path | include_code}
\end{lstlisting}
%% endfor
\subsubsection{问题三}}

\subsection{计算结果及其分析}

\subsubsection{问题一}

\begin{lstlisting}[frame=single]
>> gauss(A ,b)

ans =

    1.0000
   -1.0000
   -0.0000
    1.0000
    2.0000
    0.0000
    3.0000
    1.0000
   -1.0000
    2.0000

\end{lstlisting}

使用高斯顺序消去法求得的解为:

\[x = (1.0000, -1.0000, -0.0000, 1.0000, 2.0000, 0.0000, 3.0000, 1.0000, -1.0000, 2.0000)^T\]

和精确解相比误差不大,但高斯顺序消去法的计算复杂度达到了 $O(n^3)$,对于大规模的矩阵运算来说相当耗时,同时也不适合对增广矩阵为稀疏矩阵的线性方程组进行求解。

\subsubsection{问题二}

\begin{lstlisting}[frame=single]
>> squareRoot(A, b)

ans =

    121.1481
    -140.1127
    29.7515
    -60.1528
    10.9120
    -26.7963
    5.4259
    -2.0185

\end{lstlisting}

利用平方根法求得的解为:

\[x = (121.1481, -140.1127, 29.7515, -60.1528, 10.9120, -26.7963, 5.4259, -2.0185)^T\]

虽然系数矩阵的对角元素都大于零,原则上可以不必选择主元,但由于矩阵的数值问题较大,不选择主元的结果就是产生较大的误差,所以在求解过程中还是应该选择主元以消除此误差,提高精度。

\subsubsection{问题三}

\begin{lstlisting}[frame=single]
>> catchup(A, f)

ans =

    2.0000
    1.0000
   -3.0000
    0.0000
    1.0000
   -2.0000
    3.0000
   -0.0000
    1.0000
   -1.0000

\end{lstlisting}

使用追赶法求得的解为:

\[x= (2.0000,1.0000,-3.0000,0.0000,1.0000,-2.0000,3.0000,-0.0000,1.0000,-1.0000) ^ T\]

与精确解相比误差非常小,这说明追赶法对于解这种三对角线系数矩阵的线性方程组能在保证计算量小的同时,还表现出相当高的精度。}

\subsection{结论}

在此次课设中,我通过使用高斯顺序消元法,平方根法和追赶法分别求解了三个线性方程组,并分析了三种方法在求解线性方程组的特点。

解线性方程组,应根据它自身的特点,选择最适合的算法来进行求解,没有一种算法是最好的,只能说在某种情形下,某种算法比另外一种算法更好。
}

\end{document}
