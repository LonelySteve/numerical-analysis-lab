\subsubsection{问题一}

\begin{lstlisting}[frame=single]
>> gauss(A ,b)

ans =

    1.0000
   -1.0000
   -0.0000
    1.0000
    2.0000
    0.0000
    3.0000
    1.0000
   -1.0000
    2.0000

\end{lstlisting}

使用高斯顺序消去法求得的解为:

\[x = (1.0000, -1.0000, -0.0000, 1.0000, 2.0000, 0.0000, 3.0000, 1.0000, -1.0000, 2.0000)^T\]

和精确解相比误差不大,但高斯顺序消去法的计算复杂度达到了 $O(n^3)$,对于大规模的矩阵运算来说相当耗时,同时也不适合对增广矩阵为稀疏矩阵的线性方程组进行求解。

\subsubsection{问题二}

\begin{lstlisting}[frame=single]
>> squareRoot(A, b)

ans =

    121.1481
    -140.1127
    29.7515
    -60.1528
    10.9120
    -26.7963
    5.4259
    -2.0185

\end{lstlisting}

利用平方根法求得的解为:

\[x = (121.1481, -140.1127, 29.7515, -60.1528, 10.9120, -26.7963, 5.4259, -2.0185)^T\]

虽然系数矩阵的对角元素都大于零,原则上可以不必选择主元,但由于矩阵的数值问题较大,不选择主元的结果就是产生较大的误差,所以在求解过程中还是应该选择主元以消除此误差,提高精度。

\subsubsection{问题三}

\begin{lstlisting}[frame=single]
>> catchup(A, f)

ans =

    2.0000
    1.0000
   -3.0000
    0.0000
    1.0000
   -2.0000
    3.0000
   -0.0000
    1.0000
   -1.0000

\end{lstlisting}

使用追赶法求得的解为:

\[x= (2.0000,1.0000,-3.0000,0.0000,1.0000,-2.0000,3.0000,-0.0000,1.0000,-1.0000) ^ T\]

与精确解相比误差非常小,这说明追赶法对于解这种三对角线系数矩阵的线性方程组能在保证计算量小的同时,还表现出相当高的精度。