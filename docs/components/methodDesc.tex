\subsubsection{高斯消去法}

高斯消去法,该方法以数学家卡尔·高斯命名,但最早出现于中国古籍《九章算术》,成书于约公元前150年。

整体的思路是,将线性方程组的增广矩阵通过初等行变换得到行阶梯形矩阵,再由下至上回代计算得到线性方程组的解。

初等行变换有三种方式:

\begin{enumerate}
    \item 交换某两行
    \item 将某行乘以一个非零的数
    \item 将某行的倍数加到另一行
\end{enumerate}

在人工进行初等行变换时,可以配合使用以上三种方法来达到计算行阶梯形矩阵的目的,在计算机程序中实现时,可以仅使用最后一种方式来完成行阶梯形矩阵的计算。


\subsubsection{平方根法}

平方根法,也称为 choleskly 方法,用于将给定的系数矩阵 \(A\) 分解为
\(A = L L^T\) 的形式。这里的系数矩阵 \(A\) 是对称正定矩阵,其中的 \(L\)
是下三角矩阵。

由此可以写出以下等式: 

\begin{equation*}
    \begin{aligned}
        A = LL^T & =
        \begin{bmatrix}
            l_{11} & 0      & 0      \\
            l_{21} & l_{22} & 0      \\
            l_{31} & l_{32} & l_{33} \\
        \end{bmatrix}
        \begin{bmatrix}
            l_{11} & l_{21} & l_{31} \\
            0      & l_{22} & l_{32} \\
            0      & 0      & l_{33} \\
        \end{bmatrix} \\
                      & =
        \begin{bmatrix}
            l_{11}^2     &                             & (symmetric)                     \\
            l_{21}l_{11} & l_{21}^2 + l_{22}^2                                           \\
            l_{31}l_{11} & l_{31}l_{21} + l_{32}l_{22} & l_{31}^2 + l_{32}^2 + l_{33} ^2 \\
        \end{bmatrix}
    \end{aligned}
\end{equation*}

由此可以得到:

\begin{equation*}
    L =
    \begin{bmatrix}
        (\pm) \sqrt{a_{11}}   & 0                            & 0                               \\
        a_{21} / l_{11} &  (\pm)\sqrt{a_{22}-l_{21}^2}       & 0                               \\
        a_{31} / l_{11} & (a_{32}-l_{31}l_{21})/l_{22} &  (\pm) \sqrt{a_{33}-l_{31}^2-l_{32}^2} \\
    \end{bmatrix}
\end{equation*}

根据规律可以归纳得到:

\begin{equation*}
    l_{j,j} = (\pm) \sqrt{a_{j,j} - \sum_{k=1}^{j-1} l_{j,k}^2} \\
\end{equation*}

\[
    l_{i,j} = (a_{i,j} - \sum_{k=1}^{j-1}l_{i,k}l_{j,k}) / l_{j,j},  i > j
\]

因此,要计算${l}_{i,j}(i\neq j)$ 只需利用矩阵 $L$ 的左、上方元素的值。计算通常是以下面其中一种顺序进行的。\footnote{https://zh.wikipedia.org/wiki/Cholesky}

\begin{enumerate}
    \item Cholesky–Banachiewicz:算法从矩阵L的左上角开始,依行进行计算。
    \item Cholesky–Crout:算法从矩阵L的左上角开始,依列进行计算。
\end{enumerate}

若有需要,整个矩阵可以逐个元素计算得出,无论使用何种顺序读取。

将系数矩阵 \(A\) 分解为 \(A = L L^T\) 后,由 \(Ax = b\)
可得:\(LL^Tx = b\)。

如果进行以下分解:

\[
    \left\{\begin{matrix}
        y = L^Tx \\
        Ly = b
    \end{matrix}\right.
\]


就可以先计算 \(y\),再计算 \(x\),因为 \(L\) 是下三角矩阵,而 \(L^T\)
是上三角矩阵,这样会比直接使用 \(A\) 计算更简便。

\subsubsection{}
