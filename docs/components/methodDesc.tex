\subsubsection{}

\subsubsection{平方根法}

平方根法,也称为 choleskly 方法,用于将给定的系数矩阵 \(A\) 分解为
\(A = L L^T\) 的形式。这里的系数矩阵 \(A\) 是对称正定矩阵,其中的 \(L\)
是下三角矩阵。

因为 \(A\) 是对称的矩阵,所以设 \(A\) 为:

\[
    A =
    \begin{bmatrix}
        a_{11} & A_{21}^T \\
        A_{21} & A_{22}
    \end{bmatrix}
\]

假设:

\[
    L =
    \begin{bmatrix}
        l_{11} & 0      \\
        L_{21} & L_{22}
    \end{bmatrix}
\]

则有:

\[
    L^T =
    \begin{bmatrix}
        l_{11} & L_{21}^T \\
        0      & L_{22}^T
    \end{bmatrix}
\]

设 \(A = LL^T\) 可以得到:

\[
    \begin{aligned}
        \begin{bmatrix}
            a_{11} & A_{21}^T \\
            A_{21} & A_{22}
        \end{bmatrix}
         & =
        \begin{bmatrix}
            l_{11} & 0      \\
            L_{21} & L_{22}
        \end{bmatrix}
        \begin{bmatrix}
            l_{11} & L_{21}^T \\
            0      & L_{22}^T
        \end{bmatrix} \\
         & =
        \begin{bmatrix}
            l_{11}^2     & l_{11}L_{21}^T                  \\
            l_{11}L_{21} & L_{21}L_{21}^T + L_{22}L_{22}^T
        \end{bmatrix}
    \end{aligned}
\]

其中,未知量有 \(l_{11}\),
\(L_{21}\),\(L_{22}\),这三个未知量的求解公式如下:

\[ l_{11} = \sqrt{a_{11}} \]
\[ L_{21} = \frac{1}{l_{11}}A_{21} \]
\[ L_{22}L_{22}^T = A_{22} - L_{21}L_{21}^T \]

其中,\(l_{11}\) 和 \(L_{21}\) 显然是很容易求得的,而 \(L_{22}\)
的求解就有点意思了。通过观察可以发现,在 \(A_{22} - L_{21}L_{21}^T\)
这个式子中的 \(L_{21}\) 可以通过前面的计算得出,如果设
\(A_{22}^{'} = A_{22} - L_{21}L_{21}^T\),那么剩下的问题就是求:

\[ A_{22}^{'} = L_{22}L_{22}^T \]

显然,这同样是 Cholesky 分解,其中 \(A_{22}^{'}\)
可以通过\(A_{22} - L_{21}L_{21}^T\)求得。这个 \(L_{22}\) 代表的是 A
的右下角 n - 1 阶的子矩阵。

因此这个算法具有\textbf{递归}性质。

将系数矩阵 \(A\) 分解为 \(A = L L^T\) 后,由 \(Ax = b\)
可得:\(LL^Tx = b\)。

如果进行以下分解:


\[
    \left\{\begin{matrix}
        y = L^Tx \\
        Ly = b
    \end{matrix}\right.
\]


就可以先计算 \(y\),再计算 \(x\),因为 \(L\) 是下三角矩阵,而 \(L_T\)
是上三角矩阵,这样会比直接使用 \(A\) 计算更简便。

\subsubsection{}
