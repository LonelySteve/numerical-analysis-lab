\subsubsection{高斯消去法}

高斯消去法,该方法以数学家卡尔·高斯命名,但最早出现于中国古籍《九章算术》,成书于约公元前150年.

设有线性方程组:

\begin{equation}
    \left\{\begin{matrix}
        a_{11}x_1 + a_{12}x_2 + \cdots + a_{1n}x_n = b_1, \\
        a_{21}x_1 + a_{22}x_2 + \cdots + a_{2n}x_n = b_2, \\
        \vdots                                            \\
        a_{m1}x_1 + a_{m2}x_2 + \cdots + a_{mn}x_n = b_m.
    \end{matrix}\right.
    \label{线性方程组示例}
\end{equation}

或写成矩阵形式:

\begin{equation*}
    \begin{bmatrix}
        a_{11} & a_{12} & \cdots & a_{1n} \\
        a_{21} & a_{22} & \cdots & a_{2n} \\
        \vdots & \vdots &        & \vdots \\
        a_{m1} & a_{m2} & \cdots & a_{mn}
    \end{bmatrix}
    \begin{bmatrix}
        x_1    \\
        x_2    \\
        \vdots \\
        x_n
    \end{bmatrix}
    =
    \begin{bmatrix}
        b_1    \\
        b_2    \\
        \vdots \\
        b_m
    \end{bmatrix}
\end{equation*}

简记为 $Ax = b$.

用高斯消去法解方程组的基本思想是用逐次消去未知数的方法把源方程组 $Ax = b$ 化为与其等价的三角形方程组,而求解三角形方程组可用回代的方法求解. 换句话来说,用行的初等变换将原方程组的系数矩阵化为简单形式(上三角矩阵),从而将求解原方程组 \eqref{线性方程组示例} 的问题转化为求解简单方程组的问题,或者说,对系数矩阵 $A$ 实行一些左变换(用一些简单矩阵)将其约化为上三角矩阵.

设 $Ax = b$,其中 $A \in R^{m \times n}$.

\begin{enumerate}
    \item 如果 $a_{kk}^{k} \neq 0\quad(k=1, 2 , \cdots, n)$,则可以通过高斯消去法将 $Ax = b$ 约化为等价的三角形方程组,且计算公式为:
          \begin{enumerate}
              \item 消元计算$(k=1, 2, \cdots, n - 1)$
                    \begin{equation*}
                        \left\{\begin{matrix}
                            m_{ik} = a_{ik}^{(k)} / a_{kk}^{(k)} \quad (i = k + 1, \cdots, n),                  \\
                            a_{ij}^{(k+1)} = a_{ij}^{(k)} - m_{ik}a_{kj}^{(k)} \quad (i, j = k + 1, \cdots, n), \\
                            b_i^{(k+1)} = b_i^{(k)} - m_{ik}b_k^{(k)} \quad (i = k + 1, \cdots, n).
                        \end{matrix}\right.
                    \end{equation*}
              \item 回代计算
                    \begin{equation*}
                        \left\{\begin{matrix}
                            m_n = b_n^{(n)} / a_{nn}^{(n)}, \\
                            x_i = (b_i^{(i)} - \sum_{(j = i + 1)}^n a_{ij}^{(i)}x_j) / a_{ii} ^{(i)} \quad (i = n - 1, \cdots, 2, 1).
                        \end{matrix}\right.
                    \end{equation*}
          \end{enumerate}
    \item 如果 $A$ 为非奇异矩阵,则可通过高斯消去法(及交换两行的初等变换)将方程组 $Ax = b$ 进行约化
\end{enumerate}

\subsubsection{平方根法}

平方根法,也称为 choleskly 方法,用于将给定的系数矩阵 \(A\) 分解为
\(A = L L^T\) 的形式。这里的系数矩阵 \(A\) 是对称正定矩阵,其中的 \(L\)
是下三角矩阵。

由此可以写出以下等式:

\begin{equation*}
    \begin{aligned}
        A = LL^T & =
        \begin{bmatrix}
            l_{11} & 0      & 0      \\
            l_{21} & l_{22} & 0      \\
            l_{31} & l_{32} & l_{33} \\
        \end{bmatrix}
        \begin{bmatrix}
            l_{11} & l_{21} & l_{31} \\
            0      & l_{22} & l_{32} \\
            0      & 0      & l_{33} \\
        \end{bmatrix} \\
                 & =
        \begin{bmatrix}
            l_{11}^2     &                             & (symmetric)                     \\
            l_{21}l_{11} & l_{21}^2 + l_{22}^2                                           \\
            l_{31}l_{11} & l_{31}l_{21} + l_{32}l_{22} & l_{31}^2 + l_{32}^2 + l_{33} ^2 \\
        \end{bmatrix}
    \end{aligned}
\end{equation*}

由此可以得到:

\begin{equation*}
    L =
    \begin{bmatrix}
        (\pm) \sqrt{a_{11}} & 0                            & 0                                     \\
        a_{21} / l_{11}     & (\pm)\sqrt{a_{22}-l_{21}^2}  & 0                                     \\
        a_{31} / l_{11}     & (a_{32}-l_{31}l_{21})/l_{22} & (\pm) \sqrt{a_{33}-l_{31}^2-l_{32}^2} \\
    \end{bmatrix}
\end{equation*}

根据规律可以归纳得到:

\begin{equation*}
    l_{j,j} = (\pm) \sqrt{a_{j,j} - \sum_{k=1}^{j-1} l_{j,k}^2} \\
\end{equation*}

\[
    l_{i,j} = (a_{i,j} - \sum_{k=1}^{j-1}l_{i,k}l_{j,k}) / l_{j,j},  i > j
\]

因此,要计算${l}_{i,j}(i\neq j)$ 只需利用矩阵 $L$ 的左、上方元素的值。计算通常是以下面其中一种顺序进行的。\footnote{https://zh.wikipedia.org/wiki/Cholesky}

\begin{enumerate}
    \item Cholesky–Banachiewicz:算法从矩阵L的左上角开始,依行进行计算。
    \item Cholesky–Crout:算法从矩阵L的左上角开始,依列进行计算。
\end{enumerate}

若有需要,整个矩阵可以逐个元素计算得出,无论使用何种顺序读取。

将系数矩阵 \(A\) 分解为 \(A = L L^T\) 后,由 \(Ax = b\)
可得:\(LL^Tx = b\)。

如果进行以下分解:

\[
    \left\{\begin{matrix}
        y = L^Tx \\
        Ly = b
    \end{matrix}\right.
\]


就可以先计算 \(y\),再计算 \(x\),因为 \(L\) 是下三角矩阵,而 \(L^T\)
是上三角矩阵,这样会比直接使用 \(A\) 计算更简便。

\subsubsection{追赶法}

设系数矩阵 $A$ 为对角占优的三对角线矩阵,则有:

\begin{equation}
    A =
    \begin{bmatrix}
        b_1 & c_1                                  \\
        a_2 & b_2    & c_2                         \\
            & \ddots & \ddots  & \ddots            \\
            &        & a_{n-1} & b_{n-1} & c_{n-1} \\
            &        &         & a_n     & b_n
    \end{bmatrix}
    \begin{bmatrix}
        x_1     \\
        x_2     \\
        \vdots  \\
        x_{n-1} \\
        x_n
    \end{bmatrix}
    =
    \begin{bmatrix}
        f_1     \\
        f_2     \\
        \vdots  \\
        f_{n-1} \\
        f_n
    \end{bmatrix}
    \label{three}
\end{equation}

简记为 \textbf{$Ax=f$}. 其中当 $\left|i-j\right| > 1$ 时,$a_{ij} = 0$,且有:

\begin{enumerate}
    \item $\left|b_1\right| > \left|c_1\right| > 0$;
    \item $\left|b_i\right| \geq \left|a_i\right| + \left|c_i\right|, a_i, c_i \neq 0\quad(i=2,3,\cdots,n-1)$;
    \item $\left|b_n\right| > \left|a_n\right| > 0$.
\end{enumerate}

利用矩阵的直接三角分解法来推导解三对角线方程组 \eqref{three} 的计算公式,由于系数阵 $A$ 的特点,可以将 $A$ 分解为两个三角阵的乘积,即:

\[A=LU\]

其中 $L$ 为下三角矩阵,$U$ 为单位上三角矩阵,这种分解是可能的,设:

\begin{equation}
    A =
    \begin{bmatrix}
        b_1 & c_1                                  \\
        a_2 & b_2    & c_2                         \\
            & \ddots & \ddots  & \ddots            \\
            &        & a_{n-1} & b_{n-1} & c_{n-1} \\
            &        &         & a_n     & b_n
    \end{bmatrix}
    =
    \begin{bmatrix}
        \alpha_1                                  \\
        \gamma_2 & \alpha_2                       \\
                 & \ddots   & \ddots              \\
                 &          & \gamma_n & \alpha_n
    \end{bmatrix}
    \begin{bmatrix}
        1 & \beta_1                        \\
          & 1       & \ddots               \\
          &         & \ddots & \beta_{n-1} \\
          &         &        & 1
    \end{bmatrix}
    \label{split}
\end{equation}

其中 $\alpha_i$,$\beta_i$,$r_i$ 为待定系数,比较 \eqref{split} 两边可得

\begin{equation}
    \left.\begin{matrix}
        b_1 = \alpha_1, c_1 = \alpha_1\beta_1,                                            \\
        \alpha_i = \gamma_i, b_i = \gamma_i\beta_{i-1} + \alpha_i\quad(i = 2, \cdots, n), \\
        c_i = \alpha_i\beta_i\quad(i=2,3,\cdots,n-1).
    \end{matrix}\right\}
    \label{result}
\end{equation}

由 $\alpha_1 = b_1 \neq 0$,$\left|b_1\right| > \left|c_1\right| > 0$,$\beta_1 = c_1 / b_1$,得 $0 < \left|\beta_1\right| < 1$. 用归纳法可以证明

\[ \left|\alpha_i\right| > \left|c_i\right| \neq 0\quad(i=1,2, \cdots, n - 1)\]

即 $0 < \left|\beta_i\right| < 1$,从而可以由 \eqref{result} 求出 $\beta_i$.

求解 $Ax=f$ 等价于解两个三角形方程组

\begin{enumerate}
    \item $Ly=f$,求 $y$;
    \item $Ux=y$,求 $x$.
\end{enumerate}

从而得到解三对角线方程组的追赶法公式:

\begin{enumerate}
    \item 计算 $\{\beta_i\}$ 的递推公式 \\
          $\beta_1 = c_1 / b_1$,
          $\beta_i = c_i / (b_i - a_i\beta_{i-1})$,$\quad(i=2,3,\cdots,n-1)$;
    \item 解 $Ly = f$ \\
          $y_1 = f_1 / b_1$,
          $y_i = (f_i - a_iy_{i-1}) / (b_i - a_i\beta_{i-1})\quad(i=2,3,\cdots,n)$;
    \item 解 $Ux = y$ \\
          $x_n = y_n$,
          $x_i = y_i - \beta_ix_{i+1}\quad(i=n-1,n-2,\cdots,2,1)$.
\end{enumerate}

计算系数 $\beta_1 \rightarrow \beta_2 \rightarrow \cdots \rightarrow \beta_{n-1}$ 以及 $y_1 \rightarrow y_2 \rightarrow \cdots \rightarrow y_n$的过程称为追的过程,将计算方程组的解 $x_n \rightarrow x_{n-1} \rightarrow \cdots \rightarrow x_1$ 的过程称为赶的过程.
